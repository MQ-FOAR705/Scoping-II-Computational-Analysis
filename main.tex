\documentclass{article}
\usepackage[utf8]{inputenc}

\title{Scoping II: Computational Analysis}
\author{Sophie Wallace }
\date{August 2019}

\begin{document}

\maketitle

\section{Research Project}
Through completing the initial scoping exercise I have been able to identify numerous pains and gains that I will encounter during my research. A significant pain that I have noticed whilst being in the field and completing the scoping exercise is the disorganized and messy nature of daily field notes. Therefore I could exceedingly gain from having a tool in place that assists with qualitative data collection and organization in an ethnographic format. This would include an program or tool that organizes photo, audio, video and notes recorded in a smart phone and moved into an online storage. Once it is accessible in a desktop format the data would need to be able to manipulated, added to and organized with efficiency and accuracy. Due this being a technological task, field notes will refer to notes recorded in my phone. Anything written on paper will need to be added to manually. 

\section{Decomposition}
There are two steps to this pain and gain indicator that involves both the collection of data and the organization of data. Decomposing these steps will narrow down the scope of my project and identify the problem by exposing the components in detail. 

\subsection{Data collection}
\subsubsection{Field notes}
\begin{itemize}
\item Enter note-taking application
\item Create new text file 
\item Specify date, time and location
\item Provide context with a rich descriptive entry
\item Save file with descriptive metadata
\end{itemize}

\subsubsection{Audio Recording}
\begin{itemize}
\item Enter audio recording application
\item Connect external mic 
\item Run test for sound quality 
\item Start recording
\item State date, time, location and persons involved
\item Reiterate context, purpose and verbal consent
\item End recording once finished
\item Save file with descriptive metadata
\end{itemize}

 \subsubsection{Visual Recording}
\begin{itemize}
\item Ask permission if persons involved
\item Enter camera application
\item Take snapshot or video
\item Verbal context in video is applicable
\end{itemize}

\subsection{Data Organization}
\subsubsection{Field notes}
\begin{itemize}
\item Upload file to computer
\item Add file to main field note folder
\item Add more context to notes if needed
\item Add additional side notes and memos
\item Ensure notes have appropriate metadata
\item Save file in safe and reliable location
\end{itemize}

\subsubsection{Audio Recording}
\begin{itemize}
\item Upload file to computer
\item Add file to main audio recording folder
\item Add written context, side notes and memos
\item Ensure audio has appropriate metadata
\item Save file in safe and reliable location
\end{itemize}

\subsubsection{Visual Recording}
\begin{itemize}
\item Upload file to computer
\item Add file to main visual recording folder
\item Add written context, side notes and memos
\item Ensure recording has appropriate metadata 
\item Save file in safe and reliable location
\end{itemize}

\subsection{Pattern Recognition}
Through decomposing the steps involved in data collection and organization, it is clear to see that the most recognized pattern is in data organization. All three versions of collected data in the organize section needed to be uploaded, categorized, provided context, have appropriate  and detailed metadata and saved in a singular, safe and reliable location. Due to the repetitive nature of data organization, an efficient and effective solution should be attainable. 

\subsection{Algorithm Design}
For this pain to be resolved, an algorithm that processes data from a smart phone into a compatible, detailed and categorized desktop format needs to be established. It must also involve a step for editing and additional information to be recorded after upload and then store and save the file in a safe and reliable location. After collecting data, this design will involve:
\begin{itemize}
\item Enter file sending application
\item Select recorded data to upload
\item Add initial metadata prior to commit 
\item Select upload data to management system and storage device 
\item Data is formatted for compatibility 
\item Data is organized into categories of date and type
\item Data is sent to management system and storage platform.
\item Access system on desktop or laptop
\item Data is now categorized clearly into date, type and with initial metadata.
\item Data can be reviewed, edited and elaborated.
\item Ethnographic data is now stored in a singular, organized and secure platform.
\end{itemize}











\end{document}
